\documentclass[12pt]{article}

\usepackage{amsmath}             % This one contains many standard tools. Always include.
\usepackage{amsfonts}             % Fonts package. May be helpful for some math commands.
\usepackage{fancyheadings}	 	% Included to make headers (and footers, and margin text).
\usepackage{graphicx}            % Needed only if you want to import images.
\usepackage{mathtools}

% LaTeX has many different parameters used to set margins, and the default margins
% are pretty narrow. The geometry package allows you lots of control over the margins;
% below, we set them all to 1 in.
\usepackage[margin=1in]{geometry}


% Define a header to appear on all pages.
% Note that page numbers will appear automatically.
\pagestyle{fancyplain}
\fancypagestyle{plain}{
\renewcommand{\headrulewidth}{0.1pt}
}
% Right side of header
\rhead{\fancyplain{Problem 4}{Problem 4}}


\begin{document}

\begin{center}
\textbf{Problem 4}

Problem by \textit{The Curfew}; Writeup by Gabe
\end{center}

a.

\noindent \textbf{Algorithm:} Initialize $i := 0$. Let the store that you start in front of be known as the "origin". While you haven't found the coffee shop, repeat the following:

\begin{enumerate}

\item Travel $2^i$ spaces left.

\item Return to the origin.

\item Increment $i$ by 1.

\item Travel $2^i$ spaces right.

\item Return to the origin.

\item Increment $i$ by 1.

\end{enumerate}

\noindent \textbf{Correctness:} Imagine the coffee shop is $d$ distance away from the origin in some direction. There must be an $i$ such that $2^i \geq d$, and once we reach that $i$, our algorithm will travel $2^i$ left and right of the origin, thus passing the coffee shop and finding it. This works for all $d$.

\noindent \textbf{Cost:} Our algorithm moves $2^i$ units forward and back for all $i$ before it finds the coffee shop. At most, this is all $i$ less than or equal to $\lceil log_2(d) \rceil$, since it's possible to miss the coffee shop by not moving far enough in a direction for all $i$ up until $\lceil log_2(d) \rceil$, and then miss the coffee shop for $i = \lceil log_2(d) \rceil$ if you go in the wrong direction. Since this is the worst case analysis, we'll assume that we go in the wrong direction. This is a total of $2\sum_{i=0}^{\lceil log_2(d) \rceil} 2^i$ distance. Since the first $j$ powers of two add up to 1 fewer than the $j+1$th'd power of 2, we have the total movement as $2(2^{\lceil log_2(d) \rceil+1}-1)$, plus the final $d$ movement that takes us to the coffee shop. In other words, our total movement is $2(2^{\lceil log_2(d) \rceil+1}-1)+d$. Since $\lceil x \rceil$ is upperbounded by $x+1$, our total movement is upperbounded by $2(2^{log_2(d)+2})+d$, which is equivalent to $8(d)+d$, making our algorithm 9-competitive.

\noindent \textbf{Worst Case}: Imagine the coffee shop was $2^x+1$ right of the origin, for an even $x$, or left of the origin, for an odd $x$. Then, our algorithm iterates through the first $x+1$ values of $i$ without finding the coffee shop, since it doesn't go far enough for the first $x$ values of $i$, and goes in the wrong direction for the $x+1$st value of $i$. Each value of $i$ adds $2(2^i)$ to the total, since it goes forward and backward

b. Half of the time, perform the algorithm above, and half the time, perform the algorithm above but with directions (right and left) reversed. Correctness is trivially proven by the same reasoning used for part a.

\noindent \textbf{Cost:} Our algorithm works the exact same way as the algorithm from part a half of the time. However, we only go in the wrong direction first when $d \geq 2^i$ half of the time. The other half of the time, we go in the right direction first, and only must travel for the first $\lceil log_2(d) \rceil-1$ values of $i$ before the $d$-distance journey to the coffee shop. In this case, we only sum the first $\lceil log_2(d) \rceil-1$ powers of two, and wind up with a final distance travelled of just $2(2^{\lceil log_2(d) \rceil}-1)+d$. This is upperbounded by $5d$. Since our algorithm's running time is upperbounded by $9d$ half of the time and $5d$ the other half, in expectation our algorithm will be 7-competitive, better than the 9-competitive deterministic algorithm from part a.

\noindent \textbf{Worst Case}: not sure

\end{document}

