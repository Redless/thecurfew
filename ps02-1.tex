\documentclass{article}

\usepackage{fullpage}
\usepackage{amsmath}% http://ctan.org/pkg/amsmath

\def\coursename{CS 352: Advanced Algorithms}
\def\courseterm{Winter 2019}
\newcommand{\lecture}[1]{
   \noindent
   \begin{center}
   \framebox{
     \vbox{\hbox to 6.28in
    { {\bf \coursename} \hfill }%{\bf \courseterm}}
       \hbox to 6.28in { {\it #1\hfill}}

      }
   }
   \end{center}
   \vspace*{4mm}
}

\begin{document}

\lecture{Problem Set 2 -- Problem 1}
\textbf{Main Author: Sean Gallagher (Group-mates: )}
\\
\\
Let $OPT$ be the size of an optimal solution. Let $C_i$ be the elements selected by $Greedy$ after it has selected its first $i$ sets. Let $\alpha_{i+1}$ denote the number of new elements that will be covered by the $(i+1)$st set chosen by $Greedy$.
\\
\\
\textbf{Claim 1:} $\alpha_{i+1} \ge \frac{OPT - |C_i|}{k}$ for all $i$ where $0 \le i < k$.
\\
\\
\textbf{Proof (by contradiction):} Assume for sake of contradiction that for all $i$ where $0 \le i < k$, that $\alpha_{i+1} < \frac{OPT - |C_i|}{k}$. Notice then that $|C_i| + k\alpha_{i+1} < OPT$. 

It then follows that
\\
\\
$Greedy = |C_k|$    \indent (by definition)
\\
$=|C_i| + \alpha_{i+1} + \alpha_{i+2}+...+\alpha_{i+k}$ \indent ($Greedy$ is what was picked up to set $i$, plus what it goes on to cover)
\\
$ \le |C_i|+\alpha_{i+1}+\alpha_{i+1}+...+\alpha_{i+1}$ \indent (since $\alpha_{i} \ge \alpha_{i+j}$ for $j > 0$, because $Greedy$ picks largest covers first)
\\
$ \le |C_i|+k\alpha_{i+1}$ \indent (since there were certainly no more than $k$ $\alpha_{i+1}$'s above)
\\
$< OPT$ \indent (by what followed from our assumption)
\\
\\
We have now found that $Greedy < OPT$. We made no assumptions about the input in proving this, so it follows that $Greedy < OPT$ in \textbf{all} cases. But this is clearly not true. Consider the simple input $S=\{a\}$, $Q = \{q_1=\{a\}\}$, and $k=1$. Clearly both $Greedy$ and $OPT$ will select $q_1$ and cover the entire set, thus giving a situation where $Greedy = OPT$. So the strictness of the earlier inequality constitutes a contradiction. This proves the claim.
\\
\\
\textbf{Claim 2: } The approximation ratio is no worse than $\frac{Greedy}{OPT} \ge 1-(1-\frac{1}{k})^k$.
\\
\\
\textbf{Proof: } After a quick reordering, we want to show $Greedy \ge OPT[1-(1-\frac{1}{k})^k]$.

We will now prove the claim by showing that $|C_i| \ge OPT[1-(1-\frac{1}{k})^i]$ for all $i$ (thus including the end case where $i=k$, which is what we really seek to show). We will prove this by induction on the exponent $i$ in the inequality.

\textit{Base case: } When $i=0$, it clearly follows that $|C_0| \ge OPT[1-(1-\frac{1}{k})^0]$ as this is just saying that $0 \ge OPT[1-1] = 0$.

\textit{Inductive case: } The inductive hypothesis is that $|C_i| \ge OPT[1-(1-\frac{1}{k})^i]$. We then begin the following lines of logic from $|C_{i+1}|$:
\\
\\
$|C_{i+1}| = |C_i| + \alpha_{i+1}$ \indent (by similar logic used in parts of claim 1)
\\
$\ge |C_i|+\frac{OPT-|C_i|}{k}$ \indent (by claim 1)
\\
$=(1-\frac{1}{k})|C_i| + \frac{OPT}{k}$ \indent (by rearranging the $|C_i|$ terms)
\\
$\ge (1-\frac{1}{k})(OPT[1-(1-\frac{1}{k})^i]) + \frac{OPT}{k}$ \indent (by inductive hypothesis)
\\
$= OPT[(1-\frac{1}{k})[1-(1-\frac{1}{k})^i] + \frac{1}{k}]$ \indent (by factoring out $OPT$)
\\
$= OPT[1 - (1-\frac{1}{k})^i-\frac{1}{k}+\frac{1}{k}(1-\frac{1}{k})^i+\frac{1}{k}]$ \indent (by multiplying out terms)
\\
$= OPT[1-(1-\frac{1}{k})^i + \frac{1}{k}(1-\frac{1}{k})^i]$ \indent (by cancelling out the extra $\frac{1}{k}$'s)
\\
$= OPT[1-(1-\frac{1}{k})(1-\frac{1}{k})^i]$ \indent (by factoring)
\\
$= OPT[1-(1-\frac{1}{k})^{i+1}]$ \indent (I think this step is obvious)
\\
\\
Hence the inequality holds for all $i$, so it is true that $Greedy \ge OPT[1-(1-\frac{1}{k})^k]$, and so similarly $\frac{Greedy}{OPT} \ge 1-(1-\frac{1}{k})^k$. The algorithm is no worse than a $1-(1-\frac{1}{k})^k$ approximation. 
\\
\\
\textbf{Worst-Case Examples}\\

\vspace{3in}

\bigskip
\noindent If k = 2, we can define the problem as follows.\\

\noindent Given an interval of size n, we split n up into k parts of $\frac{n}{k}$ elements each. In this case, two halves, each with $\frac{n}{2}$ elements. Then, we can create one additional interval that has $\frac{n}{2} + 1$ elements in it and contains half of the elements in each of the original two halves. The optimal solution is picking the original two intervals, which together cover all of the original interval of size n. However, Greedy picks the larger interval on the first iteration of the loop, and then is forced to pick between two intervals of roughly size $\frac{n}{4}$, as each interval left over only contains that many new elements. Thus, in the case where k = 2 and we design the problem as described, Greedy picks $\frac{3n}{4}$ items, while OPT picks n. \\

\noindent As n gets arbitarily large, our result approaches $1 - (1 - \frac{1}{k})^{k}$ as follows:\\
 $1 - (1 - \frac{1}{k})^{k} =  1 - (1 - \frac{1}{2})^{2} =  \frac{3}{4}$.

\vspace{3in}

\end{document}

%%% Local Variables: 
%%% mode: latex
%%% TeX-master: t
%%% End: 
